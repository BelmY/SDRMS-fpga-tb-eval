%%%%%%%%%%%%%%%%%%%%%%%%%%%%%%%%%%%%%%%%%%%%%%%%%%%%%%%%%%%%%%%%%%%%%
% LaTeX Template: Project Titlepage Modified (v 0.1) by rcx
%
% Original Source: http://www.howtotex.com
% Date: February 2014
% 
% This is a title page template which be used for articles & reports.
% 
% This is the modified version of the original Latex template from
% aforementioned website.
% 
%%%%%%%%%%%%%%%%%%%%%%%%%%%%%%%%%%%%%%%%%%%%%%%%%%%%%%%%%%%%%%%%%%%%%%

\documentclass[12pt]{report}
\usepackage[a4paper]{geometry}
\usepackage[myheadings]{fullpage}
\usepackage{fancyhdr}
\usepackage{lastpage}
\usepackage{graphicx, wrapfig, subcaption, setspace, booktabs}
\usepackage[T1]{fontenc}
\usepackage[font=small, labelfont=bf]{caption}
\usepackage{fourier}
\usepackage[protrusion=true, expansion=true]{microtype}
\usepackage[english]{babel}
\usepackage{sectsty}
\usepackage{url, lipsum}
\usepackage{tgbonum}
\usepackage{hyperref}
\usepackage{xcolor}
\usepackage{booktabs}

\newcommand{\HRule}[1]{\rule{\linewidth}{#1}}
\onehalfspacing
\setcounter{tocdepth}{5}
\setcounter{secnumdepth}{5}



%-------------------------------------------------------------------------------
% HEADER & FOOTER
%-------------------------------------------------------------------------------
%\pagestyle{fancy}
%\fancyhf{}
%\setlength\headheight{15pt}
%\fancyhead[L]{Student ID: 1034511}
%\fancyhead[R]{Anglia Ruskin University}
% \fancyfoot[R]{Page \thepage\ of \pageref{LastPage}}
%-------------------------------------------------------------------------------
% TITLE PAGE
%-------------------------------------------------------------------------------

\begin{document}
{\fontfamily{serif}\selectfont
\title{  \textsc{Open Source FPGA toolchains}
		\\ [2.0cm]
		\HRule{0.5pt} \\
		\LARGE \textbf{\uppercase{Preliminary Report}
		\HRule{2pt} \\ [0.5cm]
		\normalsize March 19, 2019 \vspace*{5\baselineskip}}
		}

\date{}

\author{
		Ilias Daradimos \\ 
		}

\maketitle
\tableofcontents
\newpage

%-------------------------------------------------------------------------------
% Section title formatting
\sectionfont{\scshape}
%-------------------------------------------------------------------------------

%-------------------------------------------------------------------------------
% BODY
%-------------------------------------------------------------------------------

\section{Platforms Tested}
For the purpose of this evaluation, the platforms selected should have at least 7K LUTs and having active or under development open-source tool-chains or, as a last resort,  free toolchains.
\begin{enumerate}
    \item {Lattice ICE40}
    \newline Devboard: TinyFPGA BX
    \newline FPGA: ICE40LP8K
    \item Lattice ECP5
    \newline Devboard: ECP5 Evaluation Board
    \newline FPGA: LFE5UM5G-85F-8BG381
    \item Xilinx Artix-7
    \newline Devboard: Digilent Arty A7-100T Development Board
    \newline FPGA: XC7A100T-1CSG324C
    \item Intel Cyclone-V
    \newline Devboard: DE0-CV
    \newline FPGA: Cyclone V 5CEBA4F23C7N 

\end{enumerate}

\section{Toolchains}

\subsection{IceStorm (iCE40)}
IceStorm is used for iCE40 bitstream creation. It relies on \textbf{yosys} for synthesis and \textbf{NextPNR} for place and route.

Supports all package variants of LP1K, LP4K, LP8K and HX1K, HX4K, HX8K, LP384 and UltraPlus devices.

Does not support iCE40 LM, Ultra and UltraLite
\subsection{Trellis (ECP5)}
Project Trellis enables a fully open-source flow for ECP5 FPGAs using \textbf{Yosys} for Verilog synthesis and \textbf{nextpnr} for place and route. Project Trellis itself provides the device database and tools for bitstream creation.

The following features are currently working in the Yosys -> nextpnr -> Trellis flow.

\begin{itemize}
\item Logic slice functionality, including carries
\item Distributed RAM inside logic slices
\item All internal interconnect
\item Basic IO, including tristate, using TRELLIS\_IO primitives. LPF files and DDR inputs/outputs
\item Block RAM, using either inference in Yosys or manual instantiation of the DP16KD primitive
\item Multipliers using manual instantiation of the MULT18X18D primitive. 
\newline Inference and more advanced DSP features are not yet supported.
\item Global networks (automatically promoted and routed in nextpnr)
\item PLLs
\item Transcievers (DCUs)
\end{itemize}

\subsection{NextPNR} 
    
NextPNR is a portable FPGA place and route tool that aims to be a vendor neutral, timing driven, FOSS FPGA place and route tool.

Currently NextPNR supports:
\begin{itemize}
\item Lattice iCE40 devices through project IceStorm
\item Lattice ECP5 devices through project Trellis, currently experimental
\item "generic" back-end for user-defined architectures, currently experimental
\item Xilinx 7 Series could be supported in the future through project X-Ray 
\end{itemize}

\subsection{X-Ray (Xilinx 7)}
Project X-Ray documents the Xilinx 7-Series FPGA architecture to enable development of open-source tools.



\subsection{Symbiflow}
Symbiflow aims to be an open source flow for generating bitstreams from Verilog. 

Since it uses Yosis, icestorm, trellis, xray and nextpnr for each of the supported platforms, device support at the same level as the underlying projects.


\subsection{IceStudio}
IceStudio is a visual editor for open FPGA boards. Built on top of the Icestorm project using Apio.

It implements Graphic design -> Verilog, PCF -> Bistream -> FPGA

It supports the following devices:

\begin{itemize}
\item HX1K
\item HX8K
\item LP8K
\item UP5K
\end{itemize}

It is the easiest to setup as it handles all the toolchain and device drivers downloads and it comes as an AppImage.

\subsection{Intel Quartus Prime Lite Edition and ModelSim - Intel FPGA Starter Edition}
This is the official software from Intel. The Lite edition comes with a free license and there is a Linux version.

\subsection{Vivaldo}
This is the official software from Xilinx. The HL WEBPack edition does not require a license and there is a Linux version.
\subsection{CubicBoard}
Cubicboard claims to be an open-source FPGA project for the Cyclone-V family.

They do provide Open Hardware under Apache 2.0 license although source files are for Protel/Altium EDA.

The provided software is a VirtualBox image of Ubuntu having Intel Quartus preinstalled.

% \newpage
\section{Available features}

Based on the current project progress, implemented features are shown in tables \ref{basic-tiles}, \ref{advanced-tiles} and \ref{routing}

\begin{table}[htp]
\centering
\begin{tabular}{|l|c|c|c|}
\hline
       & \multicolumn{1}{l|}{\textbf{IceStorm}} & \multicolumn{1}{l|}{\textbf{Trellis}} & \multicolumn{1}{l|}{\textbf{X-Ray}} \\ \hline
\textbf{Logic}     & Yes     & Yes & Yes \\ \hline
\textbf{Block RAM} & Yes & Yes & Partial \\ \hline
\end{tabular}
\caption{Basic Tiles}
\label{basic-tiles}
\end{table}


\begin{table}[htp]
\centering
\begin{tabular}{|l|c|c|c|}
\hline
                     & \multicolumn{1}{l|}{\textbf{IceStorm}} & \multicolumn{1}{l|}{\textbf{Trellis}} & \multicolumn{1}{l|}{\textbf{X-Ray}} \\ \hline
\textbf{DSP}         & Yes                                    & Yes                                   & No                                  \\ \hline
\textbf{Hard Blocks} & Yes                                    & Yes                                   & No                                  \\ \hline
\textbf{Clock Tiles} & Yes                                    & Yes                                   & No                                  \\ \hline
\textbf{I/O Tiles}   & Yes                                    & Yes                                   & Partial                             \\ \hline
\end{tabular}
\caption{Advanced Tiles}
\label{advanced-tiles}
\end{table}


\begin{table}[htp]
\centering
\begin{tabular}{|l|c|c|c|}
\hline
       & \multicolumn{1}{l|}{\textbf{IceStorm}} & \multicolumn{1}{l|}{\textbf{Trellis}} & \multicolumn{1}{l|}{\textbf{X-Ray}} \\ \hline
\textbf{Logic}     & Yes     & Yes & Yes \\ \hline
\textbf{Clock} & Yes & Yes & No \\ \hline
\end{tabular}
\caption{Routing}
\label{routing}
\end{table}

%-------------------------------------------------------------------------------
% REFERENCES
%-------------------------------------------------------------------------------
\newpage
\section*{References}
Repos

https://github.com/cliffordwolf/icestorm.git

https://github.com/YosysHQ/nextpnr

https://github.com/SymbiFlow/prjtrellis

https://github.com/SymbiFlow/prjxray

https://github.com/YosysHQ/yosys

https://github.com/FPGAwars/icestudio

https://github.com/FPGAwars/apio

%[2]John W. Eaton, David Bateman, Sren Hauberg, Rik Wehbring (2015). GNU
%Octave version 4.0.0 manual: a high-level interactive language for numer-
%ical computations. Available: http://www.gnu.org/software/octave/doc/
%interpreter/. 
}
\end{document}

%-------------------------------------------------------------------------------
% SNIPPETS
%-------------------------------------------------------------------------------

%\begin{figure}[!ht]
%	\centering
%	\includegraphics[width=0.8\textwidth]{file_name}
%	\caption{}
%	\centering
%	\label{label:file_name}
%\end{figure}

%\begin{figure}[!ht]
%	\centering
%	\includegraphics[width=0.8\textwidth]{graph}
%	\caption{Blood pressure ranges and associated level of hypertension (American Heart Association, 2013).}
%	\centering
%	\label{label:graph}
%\end{figure}

%\begin{wrapfigure}{r}{0.30\textwidth}
%	\vspace{-40pt}
%	\begin{center}
%		\includegraphics[width=0.29\textwidth]{file_name}
%	\end{center}
%	\vspace{-20pt}
%	\caption{}
%	\label{label:file_name}
%\end{wrapfigure}

%\begin{wrapfigure}{r}{0.45\textwidth}
%	\begin{center}
%		\includegraphics[width=0.29\textwidth]{manometer}
%	\end{center}
%	\caption{Aneroid sphygmomanometer with stethoscope (Medicalexpo, 2012).}
%	\label{label:manometer}
%\end{wrapfigure}

%\begin{table}[!ht]\footnotesize
%	\centering
%	\begin{tabular}{cccccc}
%	\toprule
%	\multicolumn{2}{c} {Pearson's correlation test} & \multicolumn{4}{c} {Independent t-test} \\
%	\midrule	
%	\multicolumn{2}{c} {Gender} & \multicolumn{2}{c} {Activity level} & \multicolumn{2}{c} {Gender} \\
%	\midrule
%	Males & Females & 1st level & 6th level & Males & Females \\
%	\midrule
%	\multicolumn{2}{c} {BMI vs. SP} & \multicolumn{2}{c} {Systolic pressure} & \multicolumn{2}{c} {Systolic Pressure} \\
%	\multicolumn{2}{c} {BMI vs. DP} & \multicolumn{2}{c} {Diastolic pressure} & \multicolumn{2}{c} {Diastolic pressure} \\
%	\multicolumn{2}{c} {BMI vs. MAP} & \multicolumn{2}{c} {MAP} & \multicolumn{2}{c} {MAP} \\
%	\multicolumn{2}{c} {W:H ratio vs. SP} & \multicolumn{2}{c} {BMI} & \multicolumn{2}{c} {BMI} \\
%	\multicolumn{2}{c} {W:H ratio vs. DP} & \multicolumn{2}{c} {W:H ratio} & \multicolumn{2}{c} {W:H ratio} \\
%	\multicolumn{2}{c} {W:H ratio vs. MAP} & \multicolumn{2}{c} {\% Body fat} & \multicolumn{2}{c} {\% Body fat} \\
%	\multicolumn{2}{c} {} & \multicolumn{2}{c} {Height} & \multicolumn{2}{c} {Height} \\
%	\multicolumn{2}{c} {} & \multicolumn{2}{c} {Weight} & \multicolumn{2}{c} {Weight} \\
%	\multicolumn{2}{c} {} & \multicolumn{2}{c} {Heart rate} & \multicolumn{2}{c} {Heart rate} \\
%	\bottomrule
%	\end{tabular}
%	\caption{Parameters that were analysed and related statistical test performed for current study. BMI - body mass index; SP - systolic pressure; DP - diastolic pressure; MAP - mean arterial pressure; W:H ratio - waist to hip ratio.}
%	\label{label:tests}
%\end{table}%\documentclass{article}
%\usepackage[utf8]{inputenc}

%\title{Weekly Report template}
%\author{gandhalijuvekar }
%\date{January 2019}

%\begin{document}

%\maketitle

%\section{Introduction}

%\end{document}
