Version control is a vital tool for complex projects requiring the collaboration of several persons.
Version control tools provide the means for creating milestones in code, develop new or experimental features as well as revert changes that introduced bugs in an organized and structured way.

When several persons need to modify the content of the same file at the same time, all changes need to be merged into the final result. Such a task, done manually, would require tremendous effort, introduce new errors and in some cases is almost impossible. 

Version control allows several persons to interact and modify the same files and provides the means for all the changes to be merged in a functional outcome.  Parts of the content are assembled automatically and any conflicts in content can be handled efficiently.

\section{iCE40}
iCE40 projects can easily use git for version control. An example .gitignore file is presented. All files generated by Makefile are ignored.

\textbf{icestorm} and \textbf{APIO} .gitignore file contents
\begin{tcolorbox}
\begin{verbatim}
*.out
*.blif
*.bin
*.asc
*.vcd
*.rpt
abc.history
.sconsign.dblite
\end{verbatim}
\end{tcolorbox}

\section{ECP5}
\textbf{prjtrellis} projects can easily use git for version control. All files generated by Makefile are ignored

An example .gitignore file is presented
\begin{tcolorbox}
\begin{verbatim}
*.bit
*.svf
*.config
*.json
\end{verbatim}
\end{tcolorbox}

\section{Vivado}
Version control methods for Vivado are explained in  \href{https://www.xilinx.com/support/documentation/application_notes/xapp1165.pdf}{application note XAPP1165} where version control for project mode, non-project mode and IP are described.

An example .gitignore is provided by xilinx and was extended for latest Vivado version. It can be found in Vivado folder in the project repo.

\section{Quartus}
For Quartus an effort was made to determine the bare minimum of files needed in order for a project to be able to regenerated. 
A .gitignore file was created keeping all files necessary for project creation and IP regeneration. It can be found under Quartus folder in the project repository.